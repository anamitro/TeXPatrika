% Template: TeXChand Patrika
% Author: Anamitro Biswas <anamitroappu@gmail.com>
% Website: https://sites.google.com/view/anamitro
%% Copyright 2024 Anamitro Biswas
%
% This work may be distributed and/or modified under the
% conditions of the LaTeX Project Public License, either version 1.3c
% of this license or (at your option) any later version.
% The latest version of this license is in
%   https://www.latex-project.org/lppl.txt
% and version 1.3c or later is part of all distributions of LaTeX
% version 2024 or later.
%
% This work has the LPPL maintenance status `maintained'.
% 
% The Current Maintainer of this work is Anamitro Biswas.
%
% This work consists of the files texchand-patrika.tex


\documentclass[a4paper,11pt]{article}
\usepackage{xcolor}
\usepackage{newspaper}
\date{\today}
\currentvolume{1430}
\currentissue{6}
\usepackage{times}
\usepackage{graphicx}
\usepackage{multicol}
\usepackage{etoolbox}
\usepackage{picinpar}
%uasage of picinpar:
%\begin{window}[1,l,\includegraphics{},caption]xxxxx\end{window}
\usepackage{times}
\usepackage[T1]{fontenc}
\usepackage[utf8]{inputenc}
%\usepackage[nohead, nomarginpar, margin=1in, foot=.25in]{geometry}
\usepackage{lmodern}
\usepackage{fontspec}
\usepackage{setspace}
\usepackage{fancyhdr}
\usepackage{everyshi}
\usepackage[notextcomp]{stix}
%\newfontfamily\bn{Tiro Bangla}[Script=Bengali]
%\newfontfamily\enit{Tiro Bangla Italic}[Script=Latin]
%\newfontfamily\bnc{Galada}[Script=Bengali]
\usepackage{tikz}
%\makeatletter
%\renewcommand{\@chapapp}{\bn অধ্যায়}
%\makeatother
\newfontfamily\bn{TiroBangla-Regular.ttf}[Script=Bengali]
\newfontfamily\enit{IMFellEnglish-Italic.ttf}[Script=Latin]
\newfontfamily\en{IMFellEnglish-Regular.ttf}[Script=Latin]
\newfontfamily\maurya{Noto Sans Brahmi}[Script=Brahmi]
\newfontfamily\gd{IMFellEnglish-Regular.ttf}[Script=Latin]
\newfontfamily\encaps{IMFellEnglishSC-Regular.ttf}[Script=Latin]
\newfontfamily\bnc{Galada-Regular.ttf}[Script=Bengali]
\makeatletter
\def\bengalidigits#1{\expandafter\@bengali@digits #1@}
\def\@bengali@digits#1{%
  \ifx @#1
  \else
    \ifx0#1০\else\ifx1#1১\else\ifx2#1২\else\ifx3#1৩\else\ifx4#1৪\else\ifx5#1৫\else\ifx6#1৬\else\ifx7#1৭\else\ifx8#1৮\else\ifx9#1৯\fi\fi\fi\fi\fi\fi\fi\fi\fi\fi
    \expandafter\@bengali@digits
  \fi
}
\makeatother

\def\bengalinumber#1{\bengalidigits{\number#1}}
\def\bengalinumeral#1{\bengalinumber{\csname c@#1\endcsname}}

%\renewcommand{\thechapter}{\bengalinumeral{chapter}}
\renewcommand{\thesection}{\bengalinumeral{section}}
% \renewcommand{\thesubsection}{\thesection\period\bengalinumeral{subsection}}
%\renewcommand{\thesubsubsection}{\thesubsection\period\bengalinumeral{subsubsection}}
\renewcommand{\thepage}{\bengalinumeral{page}}
\renewcommand{\theenumi}{\bengalinumeral{enumi}}
\makeatother

\setmainfont{TiroBangla-Regular.ttf}
%%%%%%%%%  Front matter   %%%%%%%%%%

\pagestyle{fancy}
\lhead{Vol. 1430 Issue 6}
\chead{\bn চারণা পত্রিকা}
\rhead{\bn আশ্বিন ১৪৩০}
\lfoot{}
\cfoot{\thepage}
\rfoot{}
\renewcommand{\headrulewidth}{1pt}
%\renewcommand{\headrulewidth}{0pt}
\renewcommand{\footrulewidth}{0pt}
\usepackage{pdfpages}
%\usepackage{ifpdf,ifxetex}
%\makeatletter
%\ifpdf
  %\EveryShipout{\ifodd\c@page\else\pdfpageattr{/Rotate 180}\fi}%
%\fi
%\ifxetex
  %\EveryShipout{\ifodd\c@page\special{pdf: put @thispage << /Rotate 180 >>}%
%\fi
%}

%\fi
\makeatother
\begin{document}
%\includepdf[pages=-,signature=4,landscape]{newspaperExample.pdf}
\maketitle
\pagestyle{fancy}
\begin{multicols}{3}{

\byline{\bnc\Large মক্তব-মাদ্রাসার বাংলা}{\bn রবীন্দ্রনাথ ঠাকুর}
\begin{spacing}{1.15}
\bn বৈশাখের [১৩৩৯] প্রবাসীতে মক্তব-মাদ্রাসার বাংলা ভাষা প্রবন্ধটি পড়ে দেখলুম। আমি মূল পুস্তক পড়ি নি, ধরে নিচ্ছি প্রবন্ধ-লেখক যথোচিত প্রমাণের উপর নির্ভর করেই লিখেছেন। সাম্প্রদায়িক বিবাদে মানুষ যে কতদূর ভয়ংকর হয়ে উঠতে পারে ভারতবর্ষে আজকাল প্রতিদিনই তার দৃষ্টান্ত দেখতে পাই, কিন্তু হাস্যকর হওয়াও যে অসম্ভব নয় তার দৃষ্টান্ত এই দেখা গেল। এটাও ভাবনার কথা হতে পারত, কিন্তু সুবিধা এই যে এরকম প্রহসন নিজেকেই নিজে বিদ্রূপ করে মারে।

ভাষা মাত্রের মধ্যে একটা প্রাণধর্ম আছে। তার সেই প্রাণের নিয়ম রক্ষা করে তবেই লেখকেরা তাকে নূতন পথে চালিত করতে পারে। এ কথা মনে করলে চলবে না যে, যেমন করে হোক জোড়াতাড়া দিয়ে তার অঙ্গপ্রত্যঙ্গ বদল করা চলে। মনে করা যাক, বাংলা দেশটা মগের মুল্লুক এবং মগ রাজারা বাঙালি হিন্দু-মুসলমানের নাক-চোখের চেহারা কোনোমতে সহ্য করতে পারছে না, মনে করছে ওটাতে তাদের অমর্যাদা, তা হলে তাদের বাদশাহী বুদ্ধির কাছে একটিমাত্র অপেক্ষাকৃত সম্ভবপর পন্থা থাকতে পারে সে হচ্ছে মগ ছাড়া আর-সব জাতকে একেবারে লোপ করে দেওয়া। নতুবা বাঙালিকে বাঙালি রেখে তার নাক মুখ চোখে ছুঁচ সুতো ও শিরীষ আঠার যোগে মগের চেহারা আরোপ করবার চেষ্টা ঘোরতর দুর্দাম মগের বিচারেও সম্ভবপর বলে ঠেকতে পারে না।

এমন কোনো সভ্য ভাষা নেই যে নানা জাতির সঙ্গে নানা ব্যবহারের ফলে বিদেশী শব্দ কিছু-না-কিছু আত্মসাৎ করে নি। বহুকাল মুসলমানের সংস্রবে থাকাতে বাংলা ভাষাও অনেক পারসী শব্দ এবং কিছু কিছু আরবীও স্বভাবতই গ্রহণ করেছে। বস্তুত বাংলা ভাষা যে বাঙালি হিন্দু-মুসলমান উভয়েরই আপন তার স্বাভাবিক প্রমাণ ভাষার মধ্যে প্রচুর রয়েছে। যত বড়ো নিষ্ঠাবান হিন্দুই হোক-না কেন ঘোরতর রাগারাগির দিনেও প্রতিদিনের ব্যবহারে রাশি রাশি তৎসম ও তদ্ভব মুসলমানী শব্দ উচ্চারণ করতে তাদের কোনো সংকোচ বোধ হয় না। এমন-কি, সে-সকল শব্দের জায়গায় যদি সংস্কৃত প্রতিশব্দ চালানো যায় তা হলে পণ্ডিতী করা হচ্ছে বলে লোকে হাসবে। বাজারে এসে সহস্র টাকার নোট ভাঙানোর চেয়ে হাজার টাকার নোট ভাঙানো সহজ। সমনজারি শব্দের অর্ধেক অংশ ইংরেজি, অর্ধেক পারসী, এর জায়গায় "আহ্বান প্রচার' শব্দ সাধু সাহিত্যেও ব্যবহার করবার মতো সাহস কোনো বিদ্যাভূষণেরও হবে না। কেননা, নেহাত বেয়াড়া স্বভাবের না হলে মানুষ মার খেতে তত ভয় করে না যেমন ভয় করে লোক হাসাতে। "মেজাজটা খারাপ হয়ে আছে', এ কথা সহজেই মুখ দিয়ে বেরোয় কিন্তু যাবনিক সংসর্গ বাঁচিয়ে যদি বলতে চাই মনের গতিকটা বিকল কিংবা বিমর্ষ বা অবসাদগ্রস্ত হয়ে আছে তবে আত্মীয়দের মনে নিশ্চিত খটকা লাগবে। যদি দেখা যায় অত্যন্ত নির্জলা খাঁটি পণ্ডিতমশায় ছেলেটার ষত্বণত্ব শুদ্ধ করবার জন্যে তাকে বেদম মারছেন, তা হলে বলে থাকি, "আহা বেচারাকে মারবেন না।' যদি বলি "নিরুপায় বা নিঃসহায়কে মারবেন না' তা হলে পণ্ডিতমশায়ের মনেও করুণরসের বদলে হাস্যরসের সঞ্চার হওয়া স্বাভাবিক। নেশাখোরকে যদি মাদকসেবী বলে বসি তা হলে খামকা তার নেশা ছুটে যেতে পারে, এমন-কি, সে মনে করতে পারে তাকে একটা উচ্চ উপাধি দেওয়া হল। বদমায়েসকে দুর্বৃত্ত বললে তার চোট তেমন বেশি লাগবে না। এই শব্দগুলো যে এত জোর পেয়েছে তার কারণ বাংলা ভাষার প্রাণের সঙ্গে এদের সহজে যোগ হয়েছে।

শিশুপাঠ্য বাংলা কেতাবে গায়ের জোরে আরবীআনা পারসীআনা করাটাকেই আচারনিষ্ঠ মুসলমান যদি সাধুতা বলে জ্ঞান করেন তবে ইংরেজি স্কুলপাঠ্যের ভাষাকেও মাঝে মাঝে পারসী বা আরবী ছিটিয়ে শোধন না করেন কেন? আমিই একটা নমুনা দিতে পারি। কীট্‌সের হাইপীরিয়ন নামক কবিতাটির বিষয়টি গ্রীসীয় পৌরাণিক, তথাপি মুসলমান ছাত্রের পক্ষে সেটা যদি বর্জনীয় না হয় তবে তাতে পারসী-মিশোল করলে তার কিরকম শ্রীবৃদ্ধি হয় দেখা যাক--
\begin{large}
\color{pink}{
\enit 
Deep in the Saya-i-ghamagin of a vale,
Far sunken from the nafas-i-hayat afza-i-morn,
Far from the atshin noon and eve's one star,
Sat bamoo-i-safid Saturn Khamush as a Sang।}
\end{large}
জানি কোনো মৌলবী ছাহাব প্রকৃতিস্থ অবস্থায় ইংরেজি সাহিত্যিক ভাষার এ রকম মুসলমানীকরণের চেষ্টা করবেন না।  

\begin{window}[2,r,\includegraphics[width=2.3in]{tagore2.jpeg},\centerline{\bn রবীন্দ্রনাথ}] \bn করলেও ইংরেজি যাঁদের মাতৃভাষা এ দেশের বিদ্যালয়ে তাঁদের ভাষার এ রকম ব্যঙ্গীকরণে উচ্চাসন থেকে তাঁদের মুখ ভ্রূকুটিকুটিল হবে। আপসে যখন কথাবার্তা চালাই তখন আমাদের নিজের ভাষার সঙ্গে ইংরেজি বুলির হাস্যকর সংঘটন সর্বদাই করে থাকি; কিন্তু সে প্রহসন সাহিত্যের ভাষায় চলতি হবার কোনো আশঙ্কা নেই। জানি বাংলা দেশের গোঁড়া মক্তবেও ইংরেজি ভাষা সম্বন্ধে এ রকম অপঘাত ঘটবে না; ইংরেজের অসন্তুষ্টিই তার একমাত্র কারণ নয়। শিক্ষক জানেন পাঠ্যপুস্তকে ইংরেজিকে বিকৃতি করার অভ্যাসকে প্রশ্রয় দিলে ছাত্রদের ইংরেজি শিক্ষায় গলদ ঘটবে, তারা ঐ ভাষা সম্যকরূপে ব্যবহার করতে পারবে না। এমন অবস্থায় কীট্‌সের হাইপীরিয়নকে বরঞ্চ আগাগোড়াই ফারসীতে তর্জমা করিয়ে পড়ানো ভালো তবু তার ইংরেজিটিকে নিজের সমাজের খাতিরেও দো-আঁশলা করাটা কোনো কারণেই ভালো নয়। সেই একই কারণে ছাত্রদের নিজের খাতিরেই বাংলাটাকে খাঁটি বাংলারূপে বজায় রেখেই তাদের শেখানো দরকার। মৌলবী ছাহাব বলতে পারেন আমরা ঘরে যে বাংলা বলি সেটা ফারসী আরবী জড়ানো, সেইটাকেই মুসলমান ছেলেদের বাংলা বলে আমরা চালাব। আধুনিক ইংরেজি ভাষায় যাঁদের অ্যাংলোইণ্ডিয়ান বলে, তাঁরা ঘরে যে ইংরেজি বলেন, সকলেই জানেন সেটা আন্‌ডিফাইল্‌ড আদর্শ ইংরেজি নয়-- স্বসম্প্রদায়ের প্রতি পক্ষপাতবশত তাঁরা যদি বলেন যে, তাঁদের ছেলেদের জন্যে সেই অ্যাংলোইণ্ডিয়ানী ভাষায় পাঠ্যপুস্তক রচনা না করলে তাঁদের অসম্মান হবে, তবে সে কথাটা বিনা হাস্যে গম্ভীরভাবে নেওয়া চলবে না। বরঞ্চ এই ইংরেজি তাঁদের ছেলেদের জন্যে প্রবর্তন করলে সেইটেতেই তাঁদের অসম্মান এই কথাটাই তাঁদের অবশ্য বোঝানো দরকার হবে। হিন্দু বাঙালির সূর্যই সূর্য আর মুসলমান বাঙালির সূর্য তাম্বু, এমনতর বিদ্রূপেও যদি মনে সংকোচ না জন্মে, এতকাল একত্রবাসের পরেও প্রতিবেশীর আড়াআড়ি ধরাতলে মাথা-ভাঙাভাঙি ছাড়িয়ে যদি অবশেষে চন্দ্রসূর্যের ভাষাগত অধিকার নিয়ে অভ্রভেদী হয়ে ওঠে, তবে আমাদের ন্যাশনাল ভাগ্যকে কি কৌতুকপ্রিয় বলব, না বলব পাড়া-কুঁদুলে। পৃথিবীতে আমাদের সেই ভাগ্যগ্রহের যাঁরা প্রতিনিধি তাঁরা মুখ টিপে হাসছেন; আমরাও হাসতে চেষ্টা করি কিন্তু হাসি বুকের কাছে এসে বেধে যায়। পৃথিবীতে কম্যুনাল বিরোধ অনেক দেশে অনেক রকম চেহারা ধরেছে, কিন্তু বাংলা দেশে সেটা এই যে কিম্ভুতকিমাকার রূপ ধরল তাতে আর মান থাকে না।   
\end{window}
\end{spacing}
\closearticle


\byline{\huge \bnc বড়র পিরীতি বালুর বাঁধ}{\bn  কাজী নজরুল}
\begin{spacing}{1.15}
\bn তখন আমি আলিপুর সেন্ট্রাল জেলে রাজ-কয়েদী। অপরাধ, ছেলে খাওয়ার ঘটা দেখে রাজার মাকে একদিন রাগের চোটে ডাইনী বলে ফেলেছিলাম।… এরি মধ্যে একদিন এসিস্ট্যান্ট জেলার এসে খবর দিলেন- আবার কি মশাই, আপনি তো নোবেল প্রাইজ পেয়ে গেলেন, আপনাকে রবি ঠাকুর তাঁর ‘বসন্ত’ নাটক উৎসর্গ করেছেন।

আমার পাশেই দাঁড়য়েছিলেন আরো দু’একটি কাব্য-বাতিকগ্রস্ত রাজ-কয়েদী। আমার চেয়েও বেশী হেঁসেছিলেন সেদিন তাঁরা। আনন্দে নয়, যা নয় তাই শুনে।

কিন্তু ঐ আজগুবী গল্পও সত্য হয়ে গেল। বিশ্বকবি সত্যি সত্যিই আমার ললাটে ‘অলক্ষণের তিলক-রেখা’ এঁকে দিলেন।

অলক্ষণের তিলক-রেখাই বটে! কারণ এর পর থেকে আমার অতি অন্তরঙ্গ রাজবন্দী বন্ধুরাও আমার দিক থেকে মুখ ফিরিয়ে বসলেন। যাঁরা এতদিন আমার এক লেখাকে দশবার করে প্রশংসা করেছেন, তারাই পরে সেই লেখার পনের বার করে নিন্দা করলেন। আমার হয়ে গেল বরে শাপ!

জেলের ভিতর থেকেই শুনতে পেলাম, বাইরেও একটা বিপুল ঈর্ষা-সিন্ধু ফেনায়িত হয়ে উঠেছে। বিশ্বাস হ’ল না। বিশেষ করে যখন শুনলাম, আমারি অগ্রজপ্রতিম কোন কবিবন্ধু সেই সিন্ধু-মন্থনের অসুর-পক্ষ লীড্‌ করছেন। আমার প্রতি তাঁর অফুরন্ত স্নেহ, অপরিসীম ভালবাসার কথা শুধু যে আমরা দুজনেই জানতাম তা নয়, দেশের সকলেই জানত তাঁর গদ্যে-পদ্যে কীর্তিত আমার মহিমা-গানের ঘটা দেখে।

সত্য-সুন্দরের পূজারী বলে যাঁরা হেঁইয়ো হেঁইয়ো করে বেড়ান, তাঁদের মনও ঈর্ষায় কালো হয়ে ওঠে, শুনলে আর দুঃখ রাখবার জায়গা থাকে না।

এ খবর শুনে চোখের জল আমার চোখেই শুকিয়ে গেল। বুঝতে পারলাম শুধু আমি- বাইরের এই লাভে অন্তরের কত বড় ক্ষতি হয়ে গেল আমার। মনে মনে কেঁদে বল্‌লাম, হায় গুরুদেব! কেন আমার এত বড় ক্ষতিটা করলে!

বিশ্বকবিকে আমি শুধু শ্রদ্ধা নয়, পূজা করে এসেছি সকল হৃদয়-মন দিয়ে, যেমন ক’রে ভক্ত তার ইষ্টদেবকে পূজা করে। ছেলেবেলা থেকে তাঁর ছবি সামনে রেখে গন্ধ-ধুপ-ফুল-চন্দন দিয়ে সকাল-সন্ধা বন্দনা করেছি। এ নিয়ে কত লোকে কত ঠাট্রা-বিদ্রুপ করেছেন।

এমন কি আমার এই ভক্তির নির্মম প্রকাশ রবীন্দ্র-বিদ্বেষী কোন একজনের মাথার চাঁদিতে আজও অক্ষয় হয়ে লেখা আছে এবং এই ভক্তির বিচারের ভার একদিন আদালতের ধর্মাধিকরণের ওপরেই পড়েছিল।

আমার পরম শ্রদ্ধেয় কবি ও কথাশিল্পী মণিলাল গঙ্গোপাধ্যায় একদিন কবির সামনেই এ কথা ফাঁস করে দিলেন। কবি হেঁসে বললেন, যাক, আমার আর ভয় নেই তাহলে।

তারপর কতদিন দেখা হয়েছে, আলাপ হয়েছে। নিজের লেখা দু’চারটে কবিতা গানও শুনিয়েছি, অবশ্য কবির অনুরোধেই এবং আমার অতি সৌভাগ্যবশত তাঁর অতি প্রশসংসা লাভও করেছি কবির কাছ থেকে। সে উচ্ছ্বসিত প্রশংসায় কোনদিন এতটুকু প্রাণের দৈন্য বা মন-রাখা ভাল-বলবার চেষ্টা দেখিনি।

সষ্কোচে দূরে গিয়ে বসলে সস্নেহে কাছে ডেকে বসিয়েছেন। মনে হয়েছে, আমার পূজা সার্থক হ’ল, আমি বর পেয়ে গেলাম।

অনেক দিন তাঁর কাছে না গেলে নিজে ডেকে পাঠিয়েছেন। কতদিন তাঁর তপোবনে গিয়ে থাকবার কথা বলেছেন। হতভাগা আমি, তাঁর পায়ের তলায় বসে মন্ত্র গ্রহণের অবসর করে উঠতে পারলাম না। বনের মোষ তাড়িয়েই দিন গেল।

এই নিয়ে কতদিন তিনি আমায় কতভাবে অনুযোগ করেছেন, তুমি তলোয়ার দিয়ে দাড়ি চাঁচ্‌ছ- তোমাকে জন-সাধারণ একবারে খানায় নিয়ে গিয়ে ফেলবে- ইত্যাদি।

আমি দেখেছি, এ গৌরবে আমার মুখ যত উজ্জ্বল হয়ে উঠেছে, কোনো কোনো নাম-করা কবির মুখে কে যেন তত কালি ঢেলে দিয়েছেন। ক্রমে ক্রমে আমার জীবনের শ্রেষ্ঠ বন্ধু শুভানুধ্যায়ীরাই এমনি করে শত্রু হয়ে দাঁড়ালেন। আজ তিন চার বছর ধরে এই শুভানুধ্যায়ীরা গালি-গালাজ করেছেন আমায়, তবু তাঁদের মনের ঝাল বা প্রাণের খেদ্‌ মিটল না। বাপ রে বাপ! মানুষের গা’ল দেবার ভাষা ও তার এত কস্‌রতও থাকতে পারে, এ আমার জানা ছিল না।

ফি শনিবারে চিঠি! এবং তাতে সে কী গাড়োয়ানী রসিকতা, আর মেছোহাটা থেকে টুকে-আনা গালি! এই গালির গালিচাতে বোধ হয় আমি এ-কালের সর্ব্বশ্রেষ্ঠ শাহানশাহ্‌।

বাংলায় রেকর্ড হয়ে রইল আমায় দেওয়া এই গালির স্তূপ। কোথায় লাগে ধাপার মাঠ! ফি হপ্তায় মেল (ধাপা-মেল) বোঝাই। কিন্তু এত নিন্দাও সয়েছিল। এতদিন তবু সান্ত্বনা ছিল যে, এ হচ্ছে তন্তুবায়ের বলীবর্দ ক্রয়ের অবশ্যম্ভাবী প্রতিক্রিয়া। বাবা, তুই নখদন্তহীন নিরামিষাশী কবি, তোর কেন এ ঘোড়া-রোগ-এ স্বদেশ-প্রেমের বাই উঠল। কোথায় তুই হাঁ করে খাবি গুলবদনীর গুলিস্তানে মলয় হাওয়া, দেখবি ফুলের হাই-তোলা, গাইবি ‘আয় লো অলি কুসুম-কলি’ গান, -তা না ক’রে দিতে গেলি রাজার পেছনে খোঁচা! গেলি জেলে, টানলি ঘানি, করলি প্রয়োপবেশন, পরলি শিকল-বেড়ী, ডান্ডাবেড়ী, বইগুলোকে একধার থেকে করাতে লাগলি বাজেয়াপ্ত, এ কোন্‌ রকম রসিকতা তোর? কেনই বা এ হ্যাঙ্গাম-হুজ্জুৎ!
\begin{window}[2,r,\includegraphics[width=2.35in]{nazrul.jpg},\centerline{\bn কাজী}]
হঠাৎ একদিন দেখি ঝড় উঠেছে সাহিত্যের বেণু-বনে এবং দেখতে দেখতে সুরের বাঁশী অসুরের কোঁৎকা হয়ে উঠেছে! ছুট ছুট! যত মোলায়েম ক’রেই বেণু-বন বলি না কেন, তাতে ঝড় উঠলে যে তা চিরন্তন বাঁশ-বনই হয়ে ওঠে, তা কোন্‌ পাষণ্ড অবিশ্বাস করবে!\end{window}

বেচারী তরুণ সাহিত্য! যেন বালক অভিমন্যুকে মারতে সপ্ত মহারথীর সমাবেশ! বাইরে ছেলেমেয়ের ভিড় জমে গেল! ঘন ঘন হাততালি! বলে, ‘এই! বাঁশ-বাজি দেখতে যাবি, দৌড়ে আয়।’ কিন্তু, শুধুই কি সপ্ত মহারথীর মার! তাঁদের পেছনের পদাতিকগুলি যে আরো ভীষণ। ধূলো কাদা গোবর মাটি- কোনো রুচির বাছ-বিচার নাই, বেপরোয়া ছুঁড়ে চলেছে।

মহারথীদের মারে অমর্যাদা নেই, কিন্তু বাইরে থেকে আনা ঐ ভাড়াটে গুণ্ডাগুলোর নোংরামীতে সাহিত্যের বেনুবন যে পুকুর-পাড়ের বাঁশবাগান হয়ে উঠল!

    পুলিশের জুলুম আমার গা-সওয়া হয়ে গেছে। ওদের জুলুমের তবু একটা সীমারেখা আছে। কিন্তু সাহিত্যিক যদি জুলুম করতে শুরু করে, তার আর পারাপার নেই। এরা তখন হয়ে ওঠে টিক্‌টিকি পুলিশের চেয়েও ক্রূর, অভদ্র। যেন চাক-ভাঙ্গা ভীমরুল। জলে ডুবেও নিষ্কৃতি নেই, সেখানে গিয়েও দংশন করবে।

    পলিটিক্সের পাঁকের ভয়ে পালিয়ে গেলাম নাগালের বাইরে। মনে করলাম, যাক, এতদিনে একবার প্রাণ ভ’রে সাহিত্যের নির্মল বায়ু সেবন ক’রে অতীতের গ্লানি কাটিয়ে উঠব। ও বাবা! সাহিত্যের আসর যে পলিটিক্যাল আখড়ার চেয়েও নোংরা তা কে জানত!

কপাল, কপাল! পালিয়েও কি পার আছে? হঠাৎ একদিন সপ্ত-রথীর সপ্ত প্রহরণে চকিত হয়ে উঠলাম। ব্যাপার কি!

জানতে পালাম, আমার অপরাধ, আমি তরুণ! তরুণেরা নাকি আমায় ভালবাসে, তা’রা আমার লেখার ভক্ত। সভয়ে প্রশ্ন করলাম, আজ্ঞে, এতে আমার অপরাধটা কিসের হ’ল?

বহু কণ্ঠের হুঙ্কার উঠলো, ঐটেই তোমার অপরাধ, তুমি তরুণ এবং তরুণেরা তোমায় নিয়ে নাচে।

বললাম, আপাততঃ আপনাদের ভয়ে আজই তো বুড়ো হয়ে যেতে পারছিনে। ওর জন্য দু-দশ বছর মার খেয়েই অপেক্ষা করতে হবে। আর, যারা আমায় নিয়ে নাচে, আপনারাও তাদের নাচিয়ে ছেড়ে দিন। গোল চুকে যাবে। আমায় নিয়ে কেন টানাটানি!

আবার নেপথ্যে শোনা গেল, তুমি এই জ্যাঠা অভিমন্যুর পৃষ্ঠরক্ষী। তোমাকে মারতে পারলেই একে কতল করতে দেরী লাগবে না।

দেখাই যাক।…

এতদিন আমি উপেক্ষাবশতঃই এ ধোঁওয়া ছাড়িনি- না উনুনের, না সিগারেটের, ভেবেছিলাম, সম্রাটে সম্রাটে যুদ্ধ, দূরে দাঁড়িয়ে থাকলেই ভাল। কিন্তু হাতিতে হাতিতে লড়াই হলেও নলখাগড়ার নিস্তার নাই দেখছি। কাজেই, আমাদেরও এবার আত্মরক্ষা করতে হবে। প’ড়ে প’ড়ে মার খাওয়ায় কোন পৌরুষ নেই।

পলিটিক্সের পাঁককে যাঁরা এতদিন ঘৃণা ক’রে এসেছেন, বেণু-বনের বাঁশের প্রতি তাঁদের এই আকস্মিক আসক্তি দেখে আমারই লজ্জা করছে- বাইরের লোক কি বলছে তা না-ই বললাম।

এ-বাঁশ ছোঁড়ারও তারিফ করতে হবে। কোনো বিশেষ একজনের শির লক্ষ্য করে না ছু’ড়ে এঁরা ছুঁড়ছেন একেবারে দল লক্ষ্য করে। কারণ, তাতে লক্ষ্যভ্রষ্ট হবার লজ্জা নেই। বীর বটে! এতদিন তাই পরাস্ত মেনেই চুপ ক’রে ছিলাম। কিন্তু ঐ চুপ করে থাকি বলেই ও-পক্ষ মনে করেন, আমরা জিতে গেলাম। তাই এবার মাথা বেছে বেছেই বাঁশ ছোড়া হচ্ছে- বাণ নয়।

অবশ্য, সে বাঁশে বাঁশীর মত গোটাকতক ফুটো ক’রে সুর ফোটাবার আয়াসেরও প্রমাণ পাওয়া যায়। কিন্তু তবু তার খোঁচা আর স্থলত্বই বলে, ও বাঁশী নয়- বাঁশ।

বীণাই শোভা পায় যাঁর হাতে, তাঁকেও লাঠি ঘুরাতে দেখলে দুঃখও হয়, হাঁসিও পায়। পালোয়ানী মাতামাতিতে কে যে কম যান, তা ত বলা দুষ্কর।…

আজকের ‘বাঙ্গলার কথা’য় দেখলাম- যিনি অন্ধ ধৃতরাষ্ট্রের শত পুত্রের পক্ষ হয়ে পঞ্চ পাণ্ডবকে লাঞ্ছিত করবার সৈনাপত্য গ্রহণ করেছেন, আমাদের উভয় পক্ষের পূজ্য পিতামহ ভীষ্ম-সম সেই মহারথী কবি-গুরু এই অভিমন্যুবধে সায় দিয়েছেন। মহাভারতের ভীষ্ম এই অন্যায় যুদ্ধে সায় দেননি, বৃহত্তর ভারতের ভীষ্ম সায় দিয়েছেন- এইটেই এ যুগের পক্ষে সবচেয়ে পীড়াদায়ক।

এই অভিমন্যুর রক্ষী মনে ক’রে কবিগুরু আমায়ও বাণ নিক্ষেপ করতে ছাড়েন নি। তিনি বলেছেন, আমি কথায় কথায় ‘রক্ত’কে ‘খুন’ বলে অপরাধ করেছি।

কবির চরণে ভক্তের সশ্রদ্ধ নিবেদন, কবি ত নিজেও টুপী-পায়জামা পরেন, অথচ আমরা পরলেই তাঁর এত আক্রোশের কারণ হয়ে উঠি কেন, বুঝতে পারিনে।

এই আরবী-ফার্সি শব্দ প্রয়োগ কবিতায় শুধু আমিই করিনি। আমার বহু আগে ভারতচন্দ্র, রবীন্দ্রনাথ, সত্যেন্দ্রনাথ প্রভৃতি করে গেছেন।

আমি একটা জিনিস কিছুদিন থেকে লক্ষ ক’রে আসছি। সম্ভ্রান্ত হিন্দু বংশের অনেকেই পায়জামা-শেরওয়ানী-টুপী ব্যবহার করেন, এমন কি লুঙ্গিও বাদ যায় না। তাতে তাঁদের কেউ বিদ্রূপ করে না, তাঁদের ড্রেসের নাম হয়ে যায় তখন ওরিয়েন্টাল। কিন্তু ওইগুলোই মুসলমানেরা পরলে তাঁরা হ’য়ে যায় ‘মিঞা সাহেব’। মৌলানা সাহেব আর নারদ মুনির দাড়ির প্রতিযোগিতা হলে কে যে হারবেন বলা মুস্কিল- তবু ও নিয়ে ঠাট্রা বিদ্রূপের আর অন্ত নেই।

আমি ত টুপী পায়জামা শেরওয়ানী দাড়িকে বর্জন ক’রে চলেছি শুধু ঐ 'মিঞা সাহেব’ বিদ্রূপের ভয়ে- তবুও নিস্তার নেই।

এইবার থেকে আদালতকে না হয় বিচারালয় বলব, কিন্তু নাজির পেশ্‌কার উকিল মোক্তারকে কী বলব?

কবি-গুরুর চিরন্তনের দোহাই নিতান্ত অচল। তিনি ইটালীকে উদ্দেশ ক’রে এক কবিতা লিখেছেন। তাতে- ‘উতারো ঘোম্‌টা’ তাঁকেও ব্যবহার করতে দেখেছি। ঘোম্‌টা-খোলা শোনাই আমাদের চিরন্তন অভ্যাস। ‘উতারো ঘোম্‌টা’ আমি লিখলে হয়ত সাহিত্যিকদের কাছে অপরাধীই হতাম। কিন্তু ‘উতারো’ কথাটা যে জাতেরই হোক, ওতে এক অপূর্ব সঙ্গীত ও শ্রীর উদ্বোধন হয়েছে ও-জায়গাটায়, তা ত কেউ অস্বীকার করবে না। ঐ একটু ভালো-শোনাবার লোভেই, ঐ একটি ভিন দেশী কথার প্রয়োগে অপূর্ব রূপ ও গতি দেওয়ার আনন্দেই আমিও আরবী-ফার্সি শব্দ ব্যবহার করি। কবি-গুরুও কতদিন আলাপ-আলোচনায় এর সার্থকতার প্রশংসা করেছেন।

আজ আমাদেরও মনে হচ্ছে, আজকের রবীন্দ্রনাথ আমাদের সেই চির-চেনা রবীন্দ্রনাথ নন। তাঁর পেছনের বৈয়াকরণ পণ্ডিত এসব বলাচ্ছে তাঁকে দিয়ে।

খুন্‌’ আমি ব্যবহার করি আমার কবিতায়, মুসলমানী বা বলশেভিকী রং দেওয়ার জন্য নয়। হয়ত কবি ও-দুটোর একটারও রং আজকাল পছন্দ করছেন না, তাই এত আক্ষেপ তাঁর।

    আমি শুধু ‘খুন্‌’ নয়- বাংলায় চলতি আরো অনেক আরবী-ফার্সি শব্দ ব্যবহার করেছি আমার লেখায়। আমার দিক থেকে ওর একটা জবাবদিহি আছে। আমি মনে করি, বিশ্ব-কাব্যলক্ষ্মীরও একটা মুসলমানী ঢং আছে। ও-সাজে তাঁর শ্রীর হানি হয়েছে বলেও আমার জানা নেই। স্বার্গীয় অজিত চক্রবর্তীও এ-ঢংএর ভূয়সী প্রশংসা ক’রে গেছেন। বাঙলা কাব্য-লক্ষ্মীকে দুটো ইরানী ‘জেওর’ পরালে তাঁর জাত যায় না, বরং তাঁকে আরও ‘খুবসুরত’ই দেখায়।

আজকের করা-রক্ষ্মীর প্রায় অর্ধেক অলষ্কারই ত মুসলমানী ঢং-এর। বাইরের এ ফর্মের প্রয়োজন ও সৌকুমার্য সকল শিল্পীই স্বীকার করেন। পণ্ডিত মালবিয়া স্বীকার করতে না পারেন, কিন্তু রবীন্দ্রনাথ-অবনীন্দ্রনাথ স্বীকার করবেন।

তা ছাড়া যে ‘খুনে’র জন্য কবি-গুরু রাগ করেছেন, তা দিনরাত ব্যবহৃত হচ্ছে আমাদের কথায়, কালার বক্সে (colour box-এ) এবং তা খুন-করা, খুন-হওয়া ইত্যাদি খুনোখুনি ব্যাপারেই নয়। হৃদয়েরও ‘খুন-খারাবী’ হ’তে দেখি আজো এবং তা শুধু মুসলমান-পাড়া লেনেই হয় না। আমার একটা গানে আছে-

‘উদিবে সে রবি আমাদেরই খুনে রাঙিয়া পুনর্বার।’

এই গানটি সেদিন কবি-গুরুকে দুর্ভাগ্যক্রমে শুনিয়ে ফেলেছিলাম এবং এতেই হয়তো তাঁর ও কথার উল্লেখ। তিনি রক্তের পক্ষপাতী। অর্থাৎ ও লাইনটাকে ‘উদিবে সে রবি মোদেরি রক্তে রাঙিয়া পূর্নবার'ও করা চল্‌ত। চল্‌ত, কিন্তু ওতে ওর অর্ধেক ফোর্স কমে যেত। আমি যেখানে ‘খুন’ শব্দ ব্যবহার করেছি, সে ঐরকম ন্যাশনাল সঙ্গীতে বা রুদ্ররসের কবিতায়। যেখানে ‘রক্তধারা’ লিখবার, সেখানে জোর করে খুনধারা লিখি নাই। তাই বলে রক্ত-খারাবীও লিখি নাই, হয় রক্তারক্তি, না হয় খুন-খারাবী লিখেছি।

কবিগুরু মনে করেন, রক্তের মানেটা আরো ব্যাপক। ওটা প্রেমের কবিতাতেও চলে। চলে, কিন্তু তখন ওতে রাগ মেশাতে হয়। প্রিয়ার গালে যেমন ‘খুন’ ফোটে না, তেমনি রক্তও ফোটে না- নেহাৎ দাঁত না ফুটালে। প্রিয়ার সাথে খুনা-খুনি খেলি না, কিন্তু খুন-সুড়ি হয়ত করি।

কবিগুরু কেন, আজকালকার অনেক সাহিত্যিক ভুলে যান যে, বাংলার কাব্যলক্ষ্মীর ভক্ত অর্ধেক মুসলমান। তারা তাঁদের কাছ থেকে টুপি আর আচকান চায় না, চায় মাঝে মাঝে বেহালার সাথে সারেঙ্গীর সুর শুনতে, ফুল-বনের কোকিলের গানের বিরতিতে বাগিচায় বুল্‌বুলির সুর।

  এতেই মহাভারত অশুদ্ধ হয়ে গেল যাঁরা মনে করেন, তাঁরা সাহিত্য সভায় ভিড় না ক’রে হিন্দু-সভারই মেম্বার হন গিয়ে।

যে কবিগুরু অভিধান ছাড়া নূতন নূতন শব্দ সৃষ্টি করে ভাবীকালের জন্য আরো তিনটে অভিধানের সঞ্চয় রেখে গেলেন, তাঁর এই নূতন শব্দ-ভীতি দেখে আমরা বিস্মিত হই। মনে হয়, তাঁর আক্রোশের পেছনে অনেক কেহ এবং অনেক কিছু আছে। আরো মনে হয়, আমার শত্রু সাহিত্যকগণের অনেক দিনের অনেক মিথ্যা অভিযোগ জমে জমে ওঁর মনকে বিষিয়ে তুলেছে। নৈলে আরবী-ফার্সি শব্দের মোহ ত আমার আজকের নয়; এবং কবি-গুরুর সাথে আমার বা আমার কবিতার পরিচয়ও আজকের নয়। কই, এতদিন ত কোনো কথা উঠল না এ নিয়ে।

সবচেয়ে দুঃখ হয়, যখন দেখি কতকগুলো জোনাকিপোকা রবিলোকের বহু নিম্নে থেকেও কবিত্বের আস্ফালন করে। ভক্ত কি শুধু ঐ নোংরা লোকগুলোই, যারা রাত-দিন তাঁর কানের কাছে অন্যের কুৎসা গেয়ে তাঁর শান্ত সুন্দর মনকে নিরন্তন বিক্ষুব্ধ ক’রে তুলছে? আর, আমরা তাঁর কাছে ঘন ঘন যাইনে বলেই হয়ে গেলাম তাঁর শত্রু।

কবি-গুরুর কাছে প্রার্থনা, ঐ দৃতরাষ্ট্রের সেনাপতিত্ব তিনি করুন, দুঃখ নাই। কিন্তু ওদের প্ররোচনায় আমাদের প্রতি অহেতুক সন্দেহ পোষণ ক’রে যেন তাঁর মহিমাকে খর্ব না করেন।

সবচেয়ে কাছে যারা থাকে, দেব-মন্দিরের সেই পাণ্ডারাই দেবতার সবচেয়ে বড় ভক্ত নয়।

আরো একটা কথা। যেটা সম্বন্ধে কবি-গুরুর একটা খোলা কথা শুনতে চাই। আমাদের উদ্দেশ্য ক’রে ওঁর আজকালকার লেখাগুলোর সুর শুনে মনে হয়, আমাদের অভিশপ্ত জীবনের দারিদ্র্য নিয়েও যেন তিনি বিদ্রূপ করতে শুরু করেছেন।

আমাদের এই দুঃখকে কৃত্রিম বলে সন্দেহ করবার প্রচুর ঐশ্বর্য তাঁর আছে, জানি এবং এও জানি, তিনি জগতের সবচেয়ে বড় দুঃখ ঐ দারিদ্র ব্যতীত হয়ত আর সব দুঃখের সাথেই অল্প-বিস্তর পরিচিত। তাই এতে ব্যথা পেলেও রাগ করিনে।

কি ভীষণ দারিদ্রর সঙ্গে সংগ্রাম করে অনশনে অর্ধাশনে দিন কাটিয়ে আমাদের নতুন লেখকদের বেঁচে থাকতে হয়- লক্ষ্মীর কৃপায় কবি-গুরু কোনদিন আমাদের মত সাহিত্যিকের কুটীরে পদার্পণ করেন নি- হয়ত তাঁর মহিমা ক্ষুণ্ন হ’ত না তাতে- নৈলে দেখতে পেতেন, আমাদের জীবন-যাত্রার দৈন্য কত ভীষণ। এই দীন মলিন বেশ নিয়ে আমরা আছি দেশের একটেরে আত্মগোপন ক’রে। দেশে দেশে প্রোপাগাণ্ডা করা ত দূরের কথা, বাড়ী ছেড়ে পতে দাঁড়াতেও লজ্জা করে। কিছুতেই ছেঁড়া জামার তালিগুলোকে লুকাতে পারিনে। ভদ্র শিক্ষিতদের মাঝে ব’সে সর্বদাই মন খুঁত খুঁত করে, যেন কত বড় অপরাধ ক’রে ফেলেছি! বাইরের দৈন্য অভাব যত ভিতরে ভিতরে চাবকাতে থাকে, তত মনটা বিদ্রোহী হয়ে উঠতে থাকে।

কবি-গুরুর কাছেও শুধু ঐ দীনতার লজ্জাতেই যেতে পারিনে। ভয় হয়, এ লক্ষ্ণীছাড়া মূর্তি দেখে তাঁর দারোয়ানেরাই ঐ সুর-সভায় প্রবেশ করতে দেবে না। দীন ভক্ত তীর্থ যাত্রা করতে পারল না বলে দেবতা যদি অভিশাপ দেন, তা হ’লে এই পোড়া কপালকে দোষ দেওয়া ছাড়া কীই বা বলবার আছে!

তাঁর কাছে নিবেদন, তিনি যত ইচ্ছা বাণ নিক্ষেপ করুন, তা হয়ত সইবে, কিন্তু আমাদের একান্ত-আপনার এই দারিদ্র্য-যন্ত্রণাকে উপহাস করে যেন আর কাটাঘায়ে নুনের ছিটে না দেন। শুধু ঐ নির্মমতাটাই সইবে না।

কবি-গুরুর চরণে ভক্তের আর একটি সশ্রদ্ধ আবেদন- যদি আমাদের দোষত্রুটি হয়েই থাকে, গুরুর অধিকারে সস্নেহে তা দেখিয়ে দিন, আমরা শ্রদ্ধাবনত শিরে তাকে মেনে নিব। কিন্তু যারা শুধু কুৎসিৎ বিদ্রূপ আর গালিগালাজই করতে শিখেছে, তাঁকে তাদেরি বাহন হ’তে দেখলে আমাদের মাথা লজ্জায় বেদনায় আপনি হেঁট হয়ে যায়। বিশ্বকবি-সম্রাটের আসন-রবিলোক- কাদা ছোঁড়াছুঁড়ির বহু ঊর্ধ্বে।

কথাসাহিত্য-সম্রাট শরৎচন্দ্র শনিবারের চিঠিওয়ালাদের কাছে আমায় গালিই দেন আর যাই করুন, (জানি না, এ সংবাদ সত্য কিনা) ঐ দারিদ্র্রটুকুর অসম্মান তিনি করতে পারেন নি। অসহায় মানুষের দুঃখ-বেদনাকে তিনি এত বড় কঁরে দেখেছেন বলেই আজ তাঁর আসন রবি-লোকের কাছাকাছি গিয়ে উঠেছে।

একদিন কথাশিল্পী সুরেন্দ্র গঙ্গোপাধ্যায় মহাশয়ের কাছে গল্প শুনেছিলাম যে, শরৎচন্দ্র তাঁর বই-এর সমস্ত আয় দিয়ে পথের কুকুর’দের জন্য একটা মঠ তেরী ক’রে যাবেন। খেতে না পেয়ে পথে পথে ঘু’রে বেড়ায় যে সব হন্যে কুকুর, তারা আহার ও বাসস্থান পাবে ঐ মঠে- ফ্রি অব্ চার্জ। শরৎচন্দ্র নাকি জানতে পেরেছেন, ঐ সমস্ত পথের কুকুর পূর্বজন্মে সাহিত্যিক ছিল, ম’রে কুকুর হয়েছে। শুনলাম, ঐ মর্মে নাকি উইলও হয়ে গেছে।

ঐ গল্প শুনে আমি বারংবার শরৎচন্দ্রের উদ্দেশে মাটিতে মাথা ঠেকিয়ে প্রণাম ক’রে বলেছিলাম, ‘শরৎদা সত্যিই একজন মহাপুরুষ সত্যিই আমরা- সাহিত্যিকরা কুকুরের জাত। কুকুরের মতই আমরা না খেয়ে এবং কামড়া-কামড়ি ক’রে মরি। তাঁর সত্যিকার অতীন্দ্রিয় দৃষ্টি আছে, তিনি সাহিত্যিকদের অবতার-রূপ দেখতে পেয়েছেন।’

আজ তাই একটি মাত্র প্রার্থনা, যদি পরজন্ম থাকেই, তবে আর যেন এদেশে কবি হয়ে না জন্মাই। যদি আসি, বরং শরৎচন্দ্রের মাঠের কুকুর হয়েই আসি যেন। নিশ্চিন্তে দু’মুঠো খেয়ে বাঁচব।

\end{spacing}
\closearticle}
\end{multicols}

\begin{multicols}{2}{
\byline{\bnc\huge নজরুলের প্রতি রবীন্দ্রনাথ}{\bn পবিত্র গঙ্গোপাধ্যায়}
\begin{spacing}{1.15}
\bn নজরুল তখন আলিপুর সেন্ট্রাল জেলে কারারুদ্ধ।  আমি প্রতি সপ্তাহেই তার সঙ্গে সাক্ষাৎ করি।  আমার সঙ্গে নজরুলের নিয়মিত যোগাযোগের খবর নিশ্চয়ই রঈন্দ্রনাথের কাছে পৌঁছেছিল, নইলে মধুরায়ের গলির মেসে স্নেহভাজন বুলা মহলানবীশকে পাঠিয়ে আমাকে জোড়াসাঁকোয় ডাকিয়ে নিতেন না গুরুদেব...।  সেখানে অনুচর, তথা ভক্তজন পরিবৃত হয়ে কবিগুরু আসীন।  আমাকে দেখে প্রথমে কুশল প্রশ্ন করে নজরুলের খবর জানতে চাইলেন...বললেন, জাতির জীবনে বসন্ত এনেছে নজরুল।  তাই আমার সদ্যপ্রকাশিত ‘বসন্ত’ গীতিনাট্যখানি ওকেই উৎসর্গ করেছি।  সেখানা নিজের হাতে তাকে গিয়ে দিতে পারলে আমি খুশি হতাম।  কিন্তু আমি যখন নিজে গিয়ে তাকে দিতে পারছি না আমার হয়েই তুমি বইখানা ওকে দিও।

কবি আমাদের দিকে তাকিয়ে বললেন, বিদগ্ধ বাগ্‌বিন্যাসের যেমন মূল্য আছে সহজ সরল তীব্র ও ঋজু বাক্যের মূল্যও কিছু কম নয়।”

%এ উদ্ধৃতির প্রথম অংশটুকু আখ্যানের অন্তর্গত, আর শেষের কথাটুকু নজরুলের কবিতার প্রশংসা শুধু নয়, তাঁর শৈলীর এক ধরনের বৈধতা স্বীকার।  সে সময় বাংলায় এর বিরুদ্ধে নানা সমালোচনা আর নিন্দা উদ্গত হয়েছে।  আর দ্বিতীয় কথা হল, রবীন্দ্রনাথের কাব্য বা নাটকের উৎসর্গ, সে এক বিশাল সৌভাগ্যের ব্যাপার, হয়তো রবীন্দ্র ভক্তবৃন্দের মধ্যে এ নিয়ে তীব্র আকাঙ্ক্ষা লালিত হত, গোপন রেশারেশিও চলত।  তাই নজরুলকে ‘বসন্ত’ উৎসর্গের খবর সেখানে উপস্থিত এবং অনুপস্থিত অনেকের কাছে বজ্রপাতের মতো মনে হয়েছিল।  সেই মুহূর্তে রবীন্দ্রনাথের ‘উৎসর্গের’  গোপন প্রার্থী তাঁর ভক্তবৃন্দের মধ্যেও কম না হওয়ারই কথা।  পবিত্র গঙ্গোপাধ্যায়ের বাকি অংশটুকুও এই জন্যই আমরা পড়ব, কারণ এখানে সেই বিতর্কের ছায়া প্রবেশ করেছে, নজরুলকে তুচ্ছ করার একটা প্রয়াস ফুটে উঠছে, এবং রবীন্দ্রনাথ তাঁকে সমর্থন করে চলেছেন তা আমরা স্পষ্টই দেখতে পাই।  লক্ষ করি যে, রবীন্দ্রনাথের মতেরও সংশোধন এবং প্রতিবাদ উচ্চারিত হচ্ছে, এবং রবীন্দ্রনাথেকে নজরুলের অনুকূলে তাঁর নিজের মত খানিকটা তাঁর ব্যক্তিত্বের আধিপত্য দিয়েই রক্ষা করতে হচ্ছে।  

“কিন্তু সহজ সরল তীব্র ও ঋজু বাক্য মাত্রই কবিতা বা সাহিত্য হয়ে ওঠে না, মন্তব্য করলেন অমল হোম।

কখনও নয়।  তীব্রতাই যদি কাব্যগুণের আধার হত, তাহলে তোমার প্রতিটি কথাকেই তো কবিতা বলা যেত।  তীব্রতাও রসাত্মক হলেই কাব্য হয়ে ওঠে যেমন হয়ে উঠেছে নজরুলের বেলায়।  নজরুলকে আমি ‘বসন্ত’ গীতিনাট্য উৎসর্গ করেছি এবং উৎসর্গপত্রে তাকে ‘কবি’ বলে অভিহিত করেছি।  জানি, তোমাদের মধ্যে কেউ কেউ এটা অনুমোদন করতে পারনি।  আমার বিশ্বাস, তারা নজরুলের কবিতা না পড়েই এই মনোভাব পোষণ করছে।  আর পড়ে থাকলেও তার মধ্যে রূপ ও রসের সন্ধান করেনি, অবজ্ঞাভরে চোখ বুলিয়েছে মাত্র।

মার মার কাট কাট অসির ঝনঝনার মধ্যে রূপ ও রসের প্রক্ষেপটুকু হারিয়ে গেছে, উপস্থিত একজন মন্তব্য করলেন।

কাব্যে অসির ঝনঝনা থাকতে পারে না, এ তোমাদের আবদার বটে। সমগ্র জাতির অন্তর যখন সে সুরে বাঁধা, অসির ঝনঝনায় যখন সেখানে ঝংকার তোলে, ঐকতান সৃষ্টি হয়, তখন কাব্যে তাকে প্রকাশ করবে বই কি।  আমি যদি আজ তরুণ হতাম, আমার কলমেও ওই সুর বাজত।

(‘ভক্তরা’ জমি ছাড়ছেন না) কিন্তু তার রূপ হত ভিন্ন, আর একজনের মন্তব্য শোনা গেল।  

দুজনের প্রকাশ তো দুরকম হবেই, কিন্তু তাই বলে আমারটা নজরুলের চেয়ে ভালো হত, এমন কথা বা জোর করে বলবে কী করে ?

যাই বলুন, এ অসির ঝনঝনা জাতির মনের আবেগে ভাটা পড়ার সঙ্গে সঙ্গে নজরুলী কাব্যের জনপ্রিয়তাও মিলিয়ে যাবে, মন্তব্য এল ফরাস থেকে। 

জনপ্রিয়তা কাব্যবিচারে স্থায়ী নিরিখ নয়, কিন্তু যুগের মনকে যা প্রতিফলিত করে তা শুধু কাব্য নয় মহাকাব্য।

সবাই চুপচাপ।  প্রসঙ্গান্তর তুলে কবি জানতে চাইলেন, আমি কবে যাব নজরুলের কাছে।  আমি জানালাম, বুধবারে আমার ইন্টারভিউ-র দিন।

কে যেন দু-কপি ‘বসন্ত’ এনে দিল কবির হাতে।  তিনি একখানায় নিজের নাম দস্তখত করে আমার হাতে তুলে দিয়ে বললেন, তাকে বলো, আমি নিজের হাতে তাকে দিতে পারলাম না বলে সে যেন দুঃখ না করে।  আমি তাকে সমগ্র অন্তর দিয়ে অকুণ্ঠ আশীর্বাদ জানাচ্ছি। আর বলো, কবিতা লেখা যেন কোনো কারণেই সে বন্ধ না করে।  সৈনিক অনেক মিলবে, কিন্তু যুদ্ধে প্রেরণা যোগাবার কবিও তো চাই।
\end{spacing}
\closearticle

\byline{\bnc\large বিশ্ব-কবিসম্রাটশ্রীরবীন্দ্রনাথ ঠাকুরশ্রীশ্রীচরণারবিন্দেষু}{\bn কাজী নজরুল}
\begin{spacing}{1.15}\obeylines
\bn হে কবি হে ঋষি অন্তর্যামী আমারে করিও ক্ষমা।

পর্বতসম শত দোষত্রুটি ও চরণে হল জমা।

জানি জানি তার ক্ষমা নাই, দেব, তবু কেন মনে জাগে

তুমি মহর্ষি করিয়াছ ক্ষমা আমি চাহিবার আগে। ...

তুমি স্মরিয়াছ ভক্তেরে তব, এই গৌরবখানি

রাখিব কোথায় ভেবে নাহি পাই, আনন্দে মূক বাণী।

কাব্যলোকের বাণীবিতানের আমি কেহ নহি আর,

বিদায়ের পথে তুমি দিলে তবু কেন এ আশিস-হার ?

প্রার্থনা মোর, যদি আরবার জন্মি এ পৃথিবীতে

আসি যেন শুধু গাহন করিতে তোমার কাব্যগীতে।

\end{spacing}
\closearticle}
\end{multicols}

\end{document}
